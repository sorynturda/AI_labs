\section{Uninformed search}
\subsection{Question 1 - Finding a Fixed Food Dot using Depth-First Search}
% definire cerinta
\par Gasirea unui punct unde se afla mancare folosind Depth-First Search
\par \textbf{Depth-First Search} este un algoritm utilizat pentru explorarea grafurilor sau arborilor. Scopul acestuia este de a traversa toate nodurile unui graf, urmărind o cale cât mai adâncă înainte de a reveni și a explora căile neexplorate.
% prezentare algoritm/metoda

\begin{algorithm}
\caption{Depth-First Search}
\begin{algorithmic}[1]
\Function{DFS}{problem}
    \State stack $\gets$ Stack()
    \State visited $\gets \emptyset$
    \State node $\gets$ problem.getStartState()
    \State stack.push((node, [ ]))
    
    \While{not stack.isEmpty()}
        \State position, path $\gets$ stack.pop()
        \If{position $\notin$ visited}
            \State visited.add(position)
            \If{problem.isGoalState(position)}
                \Return path
            \EndIf
            \For{(successor, direction, cost) in problem.getSuccessors(position)}
                \If{successor $\notin$ visited}
                    \State stack.push((successor, path + [direction]))
                \EndIf
            \EndFor
        \EndIf
    \EndWhile \\
    \Return [ ]
\EndFunction
\end{algorithmic}
\end{algorithm}
% optional prezentare cod
% optional explicatii suplimentare cod

% comentarii/observatii asupra performantei/rezultatelor/complexitatii/algoritmului
\par Complexitatea algoritmului:

\begin{itemize}
    \item Timp: $O(b^d)$ unde b este factorul de ramificare (numarul mediu de succesori) si d este adancimea maxima a arborelui
    \item Spațiu: $O(b*d)$ - trebuie sa stocam nodurile de pe calea curenta plus nodurile de pe acelasi nivel 
\end{itemize}

\par Avantaje:
\begin{itemize}
	\item Implementare simpla si intuitiva
	\item Necesita mai putina memorie decat BFS deoarece exploreaza in adancime
	\item Poate gasi rapid o solutie daca aceasta se afla pe o ramura explorata devreme
\end{itemize}

\par Dezavantaje:
\begin{itemize}
	\item Nu garanteaza gasirea celui mai scurt drum
	\item Poate ramane blocata explorând cai foarte lungi/infinite daca nu se implementeaza detectia ciclurilor
\end{itemize}

\pagebreak

\subsection{Question 2 - Breadth-first search}
\par \textbf{Breadth-first search} este un algoritm utilizat pentru traversarea și căutarea în grafuri sau arbori. Spre deosebire de DFS, BFS explorează nodurile pe niveluri, adică parcurge mai întâi toate nodurile aflate la o anumită distanță de nodul de start înainte de a trece la nodurile mai îndepărtate. BFS este implementat utilizând o coadă pentru a gestiona ordinea vizitării nodurilor.
% TODO: de completat
% + q3 din project 1

\begin{algorithm}
\caption{Breadth-First Search}
\begin{algorithmic}[1]
\Function{BFS}{problem}
    \State queue $\gets$ Queue()
    \State visited $\gets \emptyset$
    \State node $\gets$ problem.getStartState()
    \State queue.push((node, [ ]))
    
    \While{not queue.isEmpty()}
        \State position, path $\gets$ queue.pop()
        \If{position $\notin$ visited}
            \State visited.add(position)
            \If{problem.isGoalState(position)}
                \Return path
            \EndIf
            \For{(successor, direction, cost) in problem.getSuccessors(position)}
                \If{successor $\notin$ visited}
                    \State queue.push((successor, path + [direction]))
                \EndIf
            \EndFor
        \EndIf
    \EndWhile \\
    \Return [ ]
\EndFunction
\end{algorithmic}
\end{algorithm}
\par Complexitatea algoritmului:

\begin{itemize}
    \item Timp: $O(b^{d+1})$ unde $b$ este factorul de ramificare și $d$ este adâncimea la care se găsește prima soluție
    \item Spațiu: $O(b^d)$ - trebuie să stocheze toate nodurile de pe nivelul curent și următorul nivel
\end{itemize}
\par BFS este optimal, spațiul fiind o mare problemă
\par Avantaje:
\begin{itemize}
	\item Garantează găsirea celei mai scurte căi până la soluție
	\item Explorează sistematic toate nodurile de pe un nivel înainte de a trece la următorul
	\item Potrivit pentru spații de căutare cu factor de ramificare mic și soluții la adâncimi mici
\end{itemize}

\par Dezavantaje:
\begin{itemize}
	\item Consumă mai multă memorie decât DFS deoarece trebuie să stocheze toate nodurile de pe un nivel
	\item Nu este potrivit pentru probleme cu ramuri infinite
	\item Poate fi ineficient pentru spații de căutare mari cu soluții la adâncimi mari
\end{itemize}

\pagebreak

\subsection{Question 3 - Uniform Cost Search}

\par Algoritmul Uniform Cost Search aplicat pe un graf găsește calea cu cel mai mic cost între un nod inițial și un nod țintă. Este o variantă a algoritmului Breadth-First Search, dar ia în considerare costurile asociate cu muchiile, fiind o metodă de cautare informată.

\begin{algorithm}
\caption{Uniform Cost Search}
\begin{algorithmic}[1]
\Function{UniformCostSearch}{problem}
    \State pq $\gets$ PriorityQueue()
    \State visited $\gets \emptyset$
    \State node $\gets$ problem.getStartState()
    \State pq.update((node, [ ], 0), 0)
    
    \While{not pq.isEmpty()}
        \State position, path, total\_cost $\gets$ pq.pop()
        \If{position $\notin$ visited}
            \State visited.add(position)
            \If{problem.isGoalState(position)}
                \Return path
            \EndIf
            \For{(successor, direction, cost) in problem.getSuccessors(position)}
                \If{successor $\notin$ visited}
                    \State pq.update((successor, path + [direction], total\_cost + cost), total\_cost + cost)
                \EndIf
            \EndFor
        \EndIf
    \EndWhile \\
    \Return [ ]
\EndFunction
\end{algorithmic}
\end{algorithm}

\par Implementare:
\begin {itemize}
	\item Foloseste o cu priorități pentru alegerea frontierei
	\item Intr-un set se pastrează toate nodurile vizitate
\end {itemize}

\par Complexitatea spatiului si a timpului paote fi mai mare decat cea a cautarii in adancime (DFS). Algoritmul este optimal, iar completitudinea este garantata daca costul fiecarei actiuni este strict pozitiv.


\pagebreak

\section{Informed search}
% q4 - q8 din project 1
\subsection{Question 4 - A* search  algorithm}

\par A* este un algoritm de căutare informată/euristică
Combină căutarea bazată pe cost (ca în UCS) cu o estimare euristică a distanței până la destinație. Este asemenea dependent de euristica folosită.

% q4 din project 1
\begin{algorithm}
\caption{A* Search}
\begin{algorithmic}[1]
\Function{AStarSearch}{problem, heuristic}
    \State start $\gets$ problem.getStartState()
    \State queue $\gets$ PriorityQueue()
    \State bestCost $\gets$ \{start: 0\}
    \State queue.push((start, []), 0)
    
    \While{not queue.isEmpty()}
        \State pos, path $\gets$ queue.pop()
        \If{problem.isGoalState(pos)}
            \Return path
        \EndIf
        
        \For{(successor, direction, cost) in problem.getSuccessors(pos)}
            \If{bestCost[pos] + cost $<$ bestCost.setdefault(successor, $\infty$)}
                \State bestCost[successor] $\gets$ cost + bestCost[pos]
                \State newCost $\gets$ bestCost[successor] + heuristic(successor, problem)
                \State queue.update((successor, path + [direction]), newCost)
            \EndIf
        \EndFor
    \EndWhile \\
	\Return [ ]
\EndFunction
\end{algorithmic}
\end{algorithm}

\par Implementare si Structuri de Date:
\begin {itemize}
	\item Coadă cu priorități (PriorityQueue) pentru selectarea nodului cu cost minim
	\item Menține un dicționar pentru costurile minime până la fiecare nod
	\item Funcția de prioritate f(n) = g(n) + h(n), unde:
	\begin {itemize}
		\item g(n) este costul real până la nodul n
		\item h(n) este euristica manhattan pentru distanța estimată până la țintă
	\end {itemize}
\end {itemize} 

\par Euristica Manhattan:

\begin {itemize}
	\item Calculează $|x_1-x_2|$ + $|y_1-y_2|$ între poziția curentă și pozitia urmatoare
	\item Este admisibilă (nu supraestimeaza costul real) si consistenta (respecta inegalitatea triunghiului)
\end {itemize}

\par Complexitate:
\begin {itemize}
	\item Timp si spațiu: $O(b^d)$ - trebuie să păstreze în memorie nodurile explorate
\end {itemize}

\par Avantaje:
\begin {itemize}
	\item Garantează găsirea drumului optim când euristica este admisibilă
	\item Mai eficient decât BFS/DFS prin folosirea euristicii pentru ghidarea căutării
	\item Explorează mai puține noduri decât algoritmii de căutare neinformată
\end {itemize}

\par Dezavantaje:
\begin {itemize}
	\item Spatiul utilizat
	\item Implementarea este mai complexă decât BFS/DFS
\end {itemize}

\pagebreak
\subsection {Question 5 - Găsirea tuturor colțurilor}
\par Pentru fiecare colț care nu a fost parcurs se calculeaza distanta manhattan pană acolo. Se alege distanța minimă, prin urmare colțul cu acea distanță se va parcurge.

\begin{algorithm}
\caption{getStartState}
\begin{algorithmic}[1]
\Function{getStartState}{}
    \State \Return (startingPosition, (False, False, False, False))
\EndFunction
\end{algorithmic}
\end{algorithm}

\par Inițial, niciun colț nu este vizitat. Tupla returnata de functia \textbf{getStartState} specifică acest lucru.

\begin{algorithm}
\caption{isGoalState}
\begin{algorithmic}[1]
\Function{isGoalState}{state}
    \State position, visitedCorners $\gets$ state
    \State \Return visitedCorners = (True, True, True, True)
\EndFunction
\end{algorithmic}
\end{algorithm}

\par \textbf{isGoalState} verifică dacă toate colțurile au fost parcurse de Pacman

\begin{algorithm}
\caption{getSuccessors}
\begin{algorithmic}[1]
\Function{getSuccessors}{state}
    \State successors $\gets$ empty list
    \State position, visitedCorners $\gets$ state
    \State x, y $\gets$ position
    \For{action in [NORTH, SOUTH, EAST, WEST]}
        \State dx, dy $\gets$ directionToVector(action)
        \State nextX $\gets$ x + dx
        \State nextY $\gets$ y + dy
        \State nextPos $\gets$ (nextX, nextY)
        \If{not walls[nextX][nextY]}
            \State newVisitedCorners $\gets$ list(visitedCorners)
            \For{i in range(4)}
                \If{corners[i] = nextPos}
                    \State newVisitedCorners[i] $\gets$ True
                \EndIf
            \EndFor
            \State cost $\gets$ 1
            \State Add ((nextPos, tuple(newVisitedCorners)), action, cost) to successors
        \EndIf
    \EndFor
    \State \Return successors
\EndFunction
\end{algorithmic}
\end{algorithm}

\pagebreak

\subsection {Question 6 - Corners Problem: Heuristic}

\begin{algorithm}
\caption{cornersHeuristic}
\begin{algorithmic}[1]
\Function{cornersHeuristic}{state, problem}
    \State position, visitedCorners $\gets$ state
    \If{visitedCorners = (True, True, True, True)}
        \State \Return 0
    \EndIf
    \State manhattanDistances $\gets$ empty list
    \For{i in range(4)}
        \If{not visitedCorners[i]}
            \State distance $\gets$ manhattanDistance(corners[i], position)
            \State Add distance to manhattanDistances
        \EndIf
    \EndFor
    \State \Return max(manhattanDistances)
\EndFunction
\end{algorithmic}
\end{algorithm}

\par Pentru fiecare colț nevizitat, se calculează distanța manhattan si se alege cea mai mare dintre toate.

\pagebreak

\subsection {Question 7 - Eating All The Dots}

\begin{algorithm}
\caption{foodHeuristic}
\begin{algorithmic}[1]
\Function{foodHeuristic}{state, problem}
    \State position, foodGrid $\gets$ state
    \State foodPositions $\gets$ foodGrid.asList()
    \State totalDistance $\gets$ 0
    
    \If{length(foodPositions) = 0}
        \State \Return 0
    \EndIf
    
    \If{length(foodPositions) = 1}
        \State \Return manhattanDistance(position, foodPositions[0])
    \EndIf
    
    \State closestFoodDistance $\gets$ $\infty$
    \For{food in foodPositions}
        \State closestFoodDistance $\gets$ min(closestFoodDistance, manhattanDistance(position, food))
    \EndFor
    
    \State closestFood $\gets$ foodPositions[0]
    \While{foodPositions is not empty}
        \State nextClosestFood $\gets$ foodPositions[0]
        \State nextMinDistance $\gets$ $\infty$
        
        \For{food in foodPositions}
            \State currentDistance $\gets$ manhattanDistance(closestFood, food)
            \If{currentDistance $<$ nextMinDistance}
                \State nextMinDistance $\gets$ currentDistance
                \State nextClosestFood $\gets$ food
            \EndIf
        \EndFor
        
        \State closestFood $\gets$ nextClosestFood
        \State totalDistance $\gets$ totalDistance + nextMinDistance
        \State Remove nextClosestFood from foodPositions
    \EndWhile
    
    \State \Return closestFoodDistance + totalDistance
\EndFunction
\end{algorithmic}
\end{algorithm}

\par La început se află distanța minimă până la cea mai apropiată mâncare folosind distanța manhattan. Apoi, se estimează o distanță totală pentru colectarea restului de mâncare. Cu alte cuvinte se construieste un drum minim ce leagă toate punctele cu mâncare: 
\begin{itemize}
	\item Se caută următoarea cea mai apropiată bucată de mâncare rămasă
	\item Se adaugă distanța până la aceasta la suma totală
	\item Se elimină bucata de mâncare găsită din lista de căutare
	\item Procesul continuă până când s-au găsit toate punctele cu mâncare
\end{itemize}
\par Rezultatul final este distanța până la cea mai apropiată mâncare + suma distanțelor minime între toate bucățile de mâncare. Euristica este admisibilă pentru că nu supraestimează niciodată costul real.

\pagebreak

\subsection {Question 8 - Suboptimal Search}

\begin{algorithm}
\caption{findPathToClosestDot}
\begin{algorithmic}[1]
\Function{findPathToClosestDot}{gameState}
    \State problem $\gets$ AnyFoodSearchProblem(gameState)
    \State \Return breadthFirstSearch(problem)
\EndFunction
\end{algorithmic}
\end{algorithm}

\par Funcția găseste cel mai scurt drum până la un punct cu mâncare. BFS va găsi garantat drumul cel mai scurt până la cea mai apropiată bucată de mâncare.


\pagebreak
\section{Adversarial search}
% q1-q3 din project2
\subsection{Question 9 - Improve the ReflexAgent} 

\par Funcția de evaluare alege optim următoarea mâncare spre care va avansa Pacman în funcție de poziția fantomelor. Se află distanța manhattan pentru mâncare si fantomă. Dacă distanța până la fantomă e foarte aproape (distanță de o mutare), Pacman nu se va apropia de fantomă (se returnează o valoare foarte mică e.g. $	- \infty$
\par Pacman va căuta tot timpul mâncarea cea mai apropiată de el. Se returnează $1 / distanta$ deoarece scorul este invers proporțional cu distanța până la mâncare. 

\begin{algorithm}
\caption{Evaluation Function}
\begin{algorithmic}[1]
\Procedure{evaluationFunction}{$self$, $currentGameState$, $action$}
    \State $minFoodDist \gets \infty$
    \For{$food \in newFood.asList()$}
        \State $minFoodDist \gets \min(minFoodDist, manhattanDistance(food, newPos))$
    \EndFor
    \For{$ghostState \in newGhostStates$}
        \If{$manhattanDistance(ghostState.getPosition(), newPos) < 2$}
            \Return $-\infty$
        \EndIf
    \EndFor
    \State \Return $1 / minFoodDist + successorGameState.getScore()$
\EndProcedure
\end{algorithmic}
\end{algorithm}
% q1 din project 2
\pagebreak
\subsection{Question 10 - Minimax}

\begin{algorithm}
\caption{getAction}
\begin{algorithmic}[1]
\Function{getAction}{gameState}
    \State maximum $\gets$ $-\infty$
    \State bestAction $\gets$ SOUTH
    \For{action in gameState.getLegalActions(0)}
        \State currentScire $\gets$ \Call{minimax}{1, 0, gameState.generateSuccessor(0, action)}
        \If{currentScire $>$ maximum}
            \State maximum $\gets$ currentScire
            \State bestAction $\gets$ action
        \EndIf
    \EndFor
    \State \Return bestAction
\EndFunction
\end{algorithmic}
\end{algorithm}

\par Algoritmul Minimax este o metodă clasică pentru luarea deciziilor în jocurile cu sumă nulă și informații perfecte, în cazul de față: Pacman.

\par Complexitatea algoritmului:

\begin{itemize}
    \item Timp: $O(b^d)$
    \item Spațiu: $O(b*d)$
\end{itemize}

\par Avantaje:
\begin {itemize}
	\item Găsește întotdeauna cea mai bună mișcare până la adâncimea dată
	\item Joacă perfect împotriva fantomelor care joacă optimal
\end {itemize}
\par Factori care influențează complexitatea
\begin{itemize}
	\item Numărul de fantome (crește factorul de ramificare $b$)
	\item Dimensiunea labirintului
\end{itemize}

\begin{algorithm}
\caption{minimax}
\begin{algorithmic}[1]
\Function{minimax}{agent, depth, gameState}
    \If{gameState.isLose() \textbf{or} gameState.isWin() \textbf{or} depth = self.depth}
        \State \Return evaluationFunction(gameState)
    \EndIf
    
    \If{agent = 0} \Comment{Pacman's turn (MAX player)}
        \State maxValue $\gets$ $-\infty$
        \For{action in gameState.getLegalActions(agent)}
            \State successor $\gets$ gameState.generateSuccessor(agent, action)
            \State value $\gets$ \Call{minimax}{1, depth, successor}
            \State maxValue $\gets$ max(maxValue, value)
        \EndFor
        \State \Return maxValue
    \Else \Comment{Ghost's turn (MIN player)}
        \State minValue $\gets$ $\infty$
        \State nextAgent $\gets$ (agent + 1) \textbf{mod} gameState.getNumAgents()
        \If{nextAgent = 0} \Comment{Back to Pacman}
            \State depth $\gets$ depth + 1
        \EndIf
        \For{action in gameState.getLegalActions(agent)}
            \State successor $\gets$ gameState.generateSuccessor(agent, action)
            \State value $\gets$ \Call{minimax}{nextAgent, depth, successor}
            \State minValue $\gets$ min(minValue, value)
        \EndFor
        \State \Return minValue
    \EndIf
\EndFunction
\end{algorithmic}
\end{algorithm}

\pagebreak
\subsection{Question 11 - $\alpha - \beta$ Pruning}  

\begin{algorithm}
\caption{getAction with Alpha-Beta Pruning}
\begin{algorithmic}[1]
\Function{getAction}{gameState}
    \State maximum $\gets$ $-\infty$
    \State $\alpha, \beta \gets -\infty, \infty$
    \State bestAction $\gets$ SOUTH
    \For{action in gameState.getLegalActions(0)}
        \State currentScore $\gets$ \Call{alfabeta}{1, 0, gameState.generateSuccessor(0, action), $\alpha$, $\beta$}
        \If{currentScore $>$ maximum}
            \State maximum $\gets$ currentScore
            \State bestAction $\gets$ action
        \EndIf
        \State $\alpha \gets$ max($\alpha$, maximum)
    \EndFor
    \State \Return bestAction
\EndFunction
\end{algorithmic}
\end{algorithm}

\par Algoritmul \textbf{Alpha-Beta Pruning} este o optimizare a algoritmului Minimax, care reduce numărul de stări evaluate fără a afecta rezultatul final. În loc să evalueze toate nodurile arborelui, Alpha-Beta ignoră ramurile care nu influențează decizia finală. Performanța optimă este atinsă atunci când nodurile sunt explorate într-o ordine bună. Dacă ordinea explorării este ideală, complexitatea se reduce la $O(b^{d/2})$. In caz contrar, performanța este similară cu Minimax.\\Alpha-Beta produce același rezultat ca Minimax fară să influențeze calitatea deciziilor.

\begin{algorithm}
\caption{alfabeta}
\begin{algorithmic}[1]
\Function{alfabeta}{agent, depth, gameState, $\alpha$, $\beta$}
    \If{gameState.isLose() \textbf{or} gameState.isWin() \textbf{or} depth = self.depth}
        \State \Return evaluationFunction(gameState)
    \EndIf
    
    \If{agent = 0} \Comment{Pacman's turn (MAX player)}
        \State value $\gets$ $-\infty$
        \For{action in gameState.getLegalActions(agent)}
            \State successor $\gets$ gameState.generateSuccessor(agent, action)
            \State value $\gets$ max(value, \Call{alfabeta}{1, depth, successor, $\alpha$, $\beta$})
            \If{value $>$ $\beta$}
                \State \Return value \Comment{Beta cutoff}
            \EndIf
            \State $\alpha \gets$ max($\alpha$, value)
        \EndFor
        \State \Return value
        
    \Else \Comment{Ghost's turn (MIN player)}
        \State value $\gets$ $\infty$
        \State nextAgent $\gets$ (agent + 1) \textbf{mod} gameState.getNumAgents()
        \If{nextAgent = 0}
            \State depth $\gets$ depth + 1
        \EndIf
        \For{action in gameState.getLegalActions(agent)}
            \State successor $\gets$ gameState.generateSuccessor(agent, action)
            \State value $\gets$ min(value, \Call{alfabeta}{nextAgent, depth, successor, $\alpha$, $\beta$})
            \If{value $<$ $\alpha$}
                \State \Return value \Comment{Alpha cutoff}
            \EndIf
            \State $\beta \gets$ min($\beta$, value)
        \EndFor
        \State \Return value
    \EndIf
\EndFunction
\end{algorithmic}
\end{algorithm}

